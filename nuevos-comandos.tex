% ------------------------------
% Lista de temas
% ------------------------------
% NOTE: Elementos en una lista
% Separar: es recomendable usar comas, no espacios
% Nombrar: NO USAR guion (-), barra baja (_) o espacio (' ')
%
% Temas de capitulos
\newcommand{\temas}{temauno,temados}

% Subtemas especificos dentro de los capitulos
\newcommand{\subtemas}{subtemauno,subsubtemauno,profundouno,subtemados,subsubtemados,profundodos}

% NOTE: Si no se necesitan hipervinculos
% Eliminar los links de temas y subtemas
% Comentar o eliminar la linea o el archivo de abajo
%
% ------------------------------
% Automatizar llamar [doc]umento [ex]terno -> \exdoc
% ------------------------------
\newcommand{\exdoc}[1]{
  \externaldocument{#1/ref-#1}
}

% [It]erar temas con [doc]umento [ex]terno -> \itexdoc
%
\newcommand{\itexdoc}[1]{
  \forcsvlist{\exdoc}{#1}
}

% Crear referencias a los documentos de los temas
%
\itexdoc{\temas}

% ------------------------------
% Automatizar crear hipervinculo -> \link
% ------------------------------
% NOTE: Se crea el hipervinculo partir de la referencia al documento externo
% por eso es importante definir el documento externo antes de esto
\newcommand{\link}[1]{
  \expandafter\newcommand\csname #1\endcsname{
    \unskip\hyperref[lbl-#1]{#1}\ignorespaces
  }
}

% [It]erar temas con \link -> \itlink
%
\newcommand{\itlink}[1]{
  \forcsvlist{\link}{#1}
}

% Crear hipervinculos
% Temas
\expandafter\itlink\expandafter{\temas}

% Subtemas
\expandafter\itlink\expandafter{\subtemas}


% ------------------------------
% Automatizar [in]put a [cap]itulos -> \incap
% ------------------------------
\newcommand{\incap}[1]{
  \input{#1/ref-#1.tex}
}

% [It]erar temas con [in]put [cap]itulo -> \itincap
%
\newcommand{\itincap}[1]{
  \forcsvlist{\incap}{#1}
}

% ------------------------------
% Lista de graficas
% Nuevo entorno flotante para graficas
% ------------------------------
% \newfloat{grafica}{htbp}{lop}
% \floatname{grafica}{Gráfica}

% ------------------------------
% Lista de ecuaciones
% Nuevo entorno flotante para ecuaciones
% ------------------------------
% \newfloat{ecuacion}{htbp}{loe}[chapter]
% \floatname{ecuacion}{Ecuación}

% Listar y crear caja (box) alrededor de una ecuacion
%
% \newcommand{\listequbox}[3]{
%   \begin{ecuacion}
%     \begin{equation}
%       \ensuremath{\boxed{#1}}
%       \label{#2}
%     \end{equation}
%     \caption{#3}
%   \end{ecuacion}
% }

% ------------------------------
% Texto reutilizable - Strings
% ------------------------------

% Asignatura
\newcommand{\asignatura}{Organización y Arquitectura de Computadoras}
\newcommand{\Asignatura}{\expandafter\MakeUppercase\expandafter{\asignatura}}

% Actividad
\newcommand{\actividad}{Investigación I}
\newcommand{\Actividad}{\expandafter\MakeUppercase\expandafter{\actividad}}

% Titulo de la actividad
\newcommand{\titulo}{Elementos Internos del Computador}
\newcommand{\Titulo}{\expandafter\MakeUppercase\expandafter{\titulo}}

% Datos estudiante 1
\newcommand{\nombreuno}{Elvis}
\newcommand{\Nombreuno}{\expandafter\MakeUppercase\expandafter{\nombreuno}}
\newcommand{\apellidouno}{Adames}
\newcommand{\Apellidouno}{\expandafter\MakeUppercase\expandafter{\apellidouno}}
% \newcommand{\cedulauno}{}
% \newcommand{\telefonouno}{}
% \newcommand{\correouno}{}

% Datos estudiante 2
\newcommand{\nombredos}{Arland}
\newcommand{\Nombredos}{\expandafter\MakeUppercase\expandafter{\nombredos}}
\newcommand{\apellidodos}{Barrera}
\newcommand{\Apellidodos}{\expandafter\MakeUppercase\expandafter{\apellidodos}}
% \newcommand{\cedulados}{}
% \newcommand{\telefonodos}{}
% \newcommand{\correodos}{}

% Datos estudiante 3
\newcommand{\nombretres}{Priscila}
\newcommand{\Nombretres}{\expandafter\MakeUppercase\expandafter{\nombretres}}
\newcommand{\apellidotres}{Ortega}
\newcommand{\Apellidotres}{\expandafter\MakeUppercase\expandafter{\apellidotres}}
% \newcommand{\cedulatres}{}
% \newcommand{\telefonotres}{}
% \newcommand{\correotres}{}

% Datos estudiante 4
\newcommand{\nombrecuatro}{Elbin}
\newcommand{\Nombrecuatro}{\expandafter\MakeUppercase\expandafter{\nombrecuatro}}
\newcommand{\apellidocuatro}{Puga}
\newcommand{\Apellidocuatro}{\expandafter\MakeUppercase\expandafter{\apellidocuatro}}
% \newcommand{\cedulacuatro}{}
% \newcommand{\telefonocuatro}{}
% \newcommand{\correocuatro}{}

% Asesor
\newcommand{\asesor}{Horacio Sandoval}
\newcommand{\Asesor}{\expandafter\MakeUppercase\expandafter{\asesor}}

% Fecha DD-MM-YYYY
% \newcommand{\dd}{26}
% \newcommand{\mm}{02}
\newcommand{\aaaa}{2025}

